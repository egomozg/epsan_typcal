\chapter{Потери активной мощности и годовые потери электроэнергии в сети}
\label{cha:70-poteri}

\section{Потери активной мощности}

Согласно справочным данным из таблицы 3.10 \cite{файбисович}, удельные среднегодовые потери мощности на корону для воздушной линии 110 кВ с одним проводом в фазе сечением $120\; \text{мм}^2$:
\[\Delta P_\textup{уд.кор.120} = 0,08\; \frac{\text{кВт}}{\text{км}}\]

Для проводов ВЛ 110 кВ сечением алюминиевой части $F_a$, отличным от $120\; \text{мм}^2$, значение среднегодовых потерь мощности на корону можно оценить по выражению:
\begin{equation}
	\Delta P_\textup{уд.кор} = \Delta P_\textup{уд.кор.120}\cdot \frac{120}{F_a}
	\label{F:korona}
\end{equation}

Тогда с учетом формулы \eqref{F:korona} среднегодовые потери мощности на корону для двухцепной ВЛ 110 кВ с проводами АС 150/24 протяженностью 84 км будут равны:
\[
\begin{split}\Delta P_\textup{кор.ИП-ПС} &= \Delta P_\textup{уд.кор.120}\cdot \frac{120}{F_a} \cdot L_\textup{ИП-ПС} \cdot n_\textup{ц.ИП-ПС} \\ &= 0,08 \cdot \frac{120}{150} \cdot 84 \cdot 2 = 0,0108\; \text{МВт}
\end{split}
\]

Потери активной мощности в сопротивлении лучей ВН, СН и НН трансформатора в режиме наибольших нагрузок:
\[\Delta P_\textup{в.нб} = 0,146\; \text{МВт};\]
\[\Delta P_\textup{с.нб} = 0,055\; \text{МВт};\]
\[\Delta P_\textup{н.нб} = 0,0214\; \text{МВт}\]

Нагрузочные потери активной мощности в обмотках трансформаторов в режиме наибольших нагрузок:
\[\Delta P_\textup{Т.нагр.нб} = \Delta P_\textup{в.нб} + \Delta P_\textup{с.нб} + \Delta P_\textup{н.нб} = 0,146 + 0,055 + 0,0214 = 0,222\; \text{МВт}\]

Эквивалентные потери активной мощности в двух трансформаторах при холостом ходе:
\[\Delta P_\textup{х} = 0,086\; \text{МВт}\]

Потери активной мощности в сопротивлении линии ИП-ПС в режиме наибольших нагрузок:
\[\Delta P_\textup{ИП-ПС.нб} = 3,24\; \text{МВт}\]

Нагрузочные потери активной мощности:
\[\Delta P_\textup{нагр.нб} = \Delta P_\textup{ИП-ПС.нб} + \Delta P_\textup{Т.нагр.нб} = 0,222 + 3,24 = 3,46\; \text{МВт}\]

Условно-постоянные потери активной мощности:
\[\Delta P_\textup{усл-пост} = \Delta P_\textup{кор.ИП-ПС} + \Delta P_\textup{х} = 0,0108 + 0,086 = 0,0968\; \text{МВт}\]

Суммарные потери активной мощности в рассматриваемой сети:
\[\Delta P_\Sigma = \Delta P_\textup{нагр.нб} + \Delta P_\textup{усл-пост} = 3,46 + 0,0968 = 3,56\; \text{МВт}\]

Суммарные потери активной мощности в рассматриваемой сети в \% от суммарной мощности нагрузки сети:
\[\Delta P_{\Sigma\%} = \frac{\Delta P_\Sigma}{P_\textup{с.нб} + P_\textup{н.нб}}\cdot 100\% = \frac{3,56}{33,0 + 22,9}\cdot 100\% = 6,37\%\]

\section{Годовые потери электроэнергии в сети}

Определим время наибольших потерь через заданное время использования наибольших нагрузок $T_\textup{нб}$ = $6820\; \frac{\text{ч}}{\text{год}}$:
\[\tau = \frac{1}{3} \cdot T_\textup{нб} + \frac{2}{3} \cdot \frac{T_\textup{нб}^2}{8760} = \frac{1}{3} \cdot 6820 + \frac{2}{3} \cdot \frac{6820^2}{8760} = 5813,1\; \frac{\text{ч}}{\text{год}}\]

Нагрузочные годовые потери электроэнергии:
\[\Delta \text{Э}_\textup{нагр} = \Delta P_\textup{нагр.нб}\cdot \tau = 3,46\cdot 5813,1 = 20113,3\; \frac{\text{МВт}\cdot \text{ч}}{\text{год}}\]

Условно-постоянные годовые потери электроэнергии:
\[\Delta \text{Э}_\textup{усл-пост} = \Delta P_\textup{усл-пост}\cdot T_\textup{год} = 0,0968 \cdot 8760 = 848,0\; \frac{\text{МВт}\cdot \text{ч}}{\text{год}}\]

Суммарные годовые потери электроэнергии в рассматриваемой сети:
\[\Delta \text{Э}_\Sigma = \Delta \text{Э}_\textup{нагр} + \Delta \text{Э}_\textup{усл-пост} = 20113,3 + 848,0 = 20961,3\; \frac{\text{МВт}\cdot \text{ч}}{\text{год}}\]

Суммарная отпущенная потребителям электроэнергия с шин среднего и низшего напряжений ПС:
\[\text{Э}_\textup{отп} = (P_\textup{с.нб} + P_\textup{н.нб})\cdot T_\textup{нб} = (33,0 + 22,9) \cdot 6820 = 381238\; \frac{\text{МВт}\cdot \text{ч}}{\text{год}}\]

Суммарные годовые потери электроэнергии в рассматриваемой сети в \% от отпущенной потребителям электроэнергии:
\[\Delta \text{Э}_{\Delta\%} = \frac{\Delta \text{Э}_\Sigma}{\text{Э}_\textup{отп}}\cdot 100\% = \frac{20961,3}{381238} \cdot 100\% = 5,50\%\]

Результаты расчета показали, что выраженные в процентах суммарные потери активной мощности в электрической сети превышают годовые потери электроэнергии, т.е. $\Delta \text{Э}_{\Sigma\%} < \Delta P_{\Sigma\%}$ соблюдается $(5,50\% < 6,37\%)$.