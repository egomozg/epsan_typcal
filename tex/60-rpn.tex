\chapter{Оценка достаточности регулировочных диапазонов устройств РПН трансформаторов на подстанции}
\label{cha:60-rpn}

Из расчёта режимов наименьших и наибольших нагрузок, а так же послеаварийного режима, известны значения напряжений на шинах среднего и низшего напряжений, приведённых к стороне высшего.

Режим наибольших нагрузок:
\[U_\textup{н.нб}^{'} = 106,4\; \text{кВ}; \qquad U_\textup{с.нб}^{'} = 107,4\; \text{кВ}\]

Режим наименьших нагрузок:
\[U_\textup{н.нм}^{'} = 99,6\; \text{кВ}; \qquad U_\textup{с.нм}^{'} = 100,0\; \text{кВ}\]

Послеаварийный режим:
\[U_\textup{н.п.ав}^{'} = 92,3\; \text{кВ}; \qquad U_\textup{с.п.ав}^{'} = 93,5\; \text{кВ}\]

Номинальные напряжения обмоток ВН, СН, НН трансформатора находим по табл. \ref{tab:tabl4}:
\[U_\textup{ВН.ном} = 115\; \text{кВ}; \qquad U_\textup{СН.ном} = 38,5\; \text{кВ}; \qquad U_\textup{НН.ном} = 11\; \text{кВ}\]

Диапазон регулирования РПН на стороне ВН: $\pm 9 \times 1,78 \%$

Диапазон регулирования ПБВ на стороне СН: $\pm 2 \times 2,5 \%$

Определим желаемый уровень напряжения на шинах СН $ 35\; \text{кВ}$.

Желаемое напряжение в режиме наибольших нагрузок:
\[U_\textup{с.нб}^\textup{жел} = 1,1 \cdot U_\textup{с.ном} = 1,1 \cdot 35 = 38,5\; \text{кВ}\]

Желаемое напряжение в режиме наименьших нагрузок:
\[U_\textup{с.нм}^\textup{жел} = 1,05 \cdot U_\textup{с.ном} = 1,05 \cdot 35 = 36,8\; \text{кВ}\]

В соответствии с п.1.2.23 ПУЭ \cite{пуэ7} устройства регулирования напряжения должны обеспечивать поддержание напряжения на шинах напряжением 3-20 кВ электростанций и подстанций, к которым присоединены распределительные сети, в пределах не ниже 105 \% номинального в период наибольших нагрузок и не выше 100 \% номинального в период наименьших нагрузок этих сетей.

Определим желаемый уровень напряжения на шинах НН $ 10\; \text{кВ}$.

Желаемое напряжение в режиме наибольших нагрузок:
\[U_\textup{н.нб}^\textup{жел} \geq 1,05 \cdot U_\textup{н.ном} = 1,05 \cdot 10 = 10,5\; \text{кВ}\]

Желаемое напряжение в режиме наименьших нагрузок:
\[U_\textup{н.нм}^\textup{жел} \leq 1,0 \cdot U_\textup{н.ном} = 1,0 \cdot 10 = 10,0\; \text{кВ}\]

\section{Режим наибольших нагрузок}

\textbf{Сторона НН}
 
Желаемое напряжение на шинах НН вычисляется по формуле:
\begin{eqndesc}
	\begin{equation}
		U_\textup{н}^\textup{жел} = \frac{U_\text{н}^{'}\cdot U_\textup{НН.ном}}{U_\text{ВН.ном}\cdot \left(1 + n_\textup{отв}^\textup{жел} \cdot \frac{\Delta U_\textup{отв}}{100\%}\right)},
		\label{F:U_zhelNN}
	\end{equation}

где $n_\textup{отв}^\textup{жел}$ "--- желаемое ответвление регулируемой части обмотки ВН; \\
$\Delta U_\textup{отв}$ "--- ступень регулирования одного ответвления в процентах.
\end{eqndesc}

Выразим из формулы \eqref{F:U_zhelNN} ответвление регулируемой части обмотки ВН, обеспечивающее желаемое напряжение на шинах НН в режиме наибольших нагрузок:
\[n_\textup{отв}^\textup{жел} \leq \left(\frac{U_\textup{н.нб}^{'}\cdot U_\textup{НН.ном}}{U_\textup{н.нб}^\textup{жел}\cdot U_\textup{ВН.ном} - 1}\right) \cdot \frac{100\%}{\Delta U_\textup{отв}} = \left(\frac{106,4\cdot 11}{10,5\cdot 115} - 1 \right) \cdot \frac{100\%}{1,78\%} = -1,73\]

Полученное значение округляем до целого числа с учетом знака неравенства и предельного количества ответвлений $(-9 \leq n_\textup{отв}^\textup{действ} \leq +9)$, то есть $n_\textup{отв}^\textup{действ} = -2$.

Рассчитаем действительное напряжение на шинах НН в режиме наибольших нагрузок:
\[U_\textup{н.нб}^\textup{действ} = \frac{U_\textup{н.нб}^{'}\cdot U_\textup{НН.ном}}{U_\textup{ВН.ном}\cdot \left(1 + n_\textup{отв}^\textup{действ} \cdot \frac{\Delta U_\textup{отв}}{100\%}\right)} = \frac{106,4\cdot 11}{115 \cdot \left(1 - 2 \cdot \frac{1,78\%}{100\%}\right)} = 10,55\; \text{кВ}\]

\textbf{Сторона СН}

Желаемое напряжение на шинах СН находится по формуле:
\begin{eqndesc}
	\begin{equation}
		U_\textup{с}^\textup{жел} = \frac{U_\textup{с}^{'}\cdot U_\textup{СН.ном}}{U_\textup{ВН.ном}\cdot \left(1 + n_\textup{отв}^\textup{действ} \cdot \frac{\Delta U_\textup{отв}}{100\%}\right)} \cdot \left(1 + n_\textup{отв.пбв}^\textup{жел} \cdot \frac{\Delta U_\textup{отв.пбв}}{100\%}\right),
		\label{F:U_zhelSN}
	\end{equation}

	где $n_\textup{отв.пбв}^\textup{жел}$ "--- желаемое ответвление регулируемой части обмотки СН; \\
	$\Delta U_\textup{отв.пбв}$ "--- ступень регулирования одного ответвления устройства ПБВ в процентах.
\end{eqndesc}

Выразим из формулы \eqref{F:U_zhelSN} ответвление регулируемой части обмотки, обеспечивающее желаемое напряжение на шинах СН в режиме наибольших нагрузок:
\[
\begin{split}
n_\textup{отв.пбв}^\textup{жел} &= \left(\frac{U_\textup{с.нб}^\textup{жел}\cdot U_\textup{ВН.ном}\cdot \left(1 + n_\textup{отв}^\textup{действ}\cdot \frac{\Delta U_\textup{отв}}{100\%}\right)}{U_\textup{с.нб}^{'}\cdot U_\textup{СН.ном}} - 1\right) \cdot \frac{100\%}{\Delta U_\textup{отв.пбв}} = \\
&= \left(\frac{38,5 \cdot 115 \cdot \left(1 - 2 \cdot \frac{1,78\%}{100\%}\right)}{107,4 \cdot 38,5} - 1\right) \cdot \frac{100\%}{2,5\%} = 1,31
\end{split}
\]

Полученное значение округляем до целого числа с учетом пределов ответвлений ПБВ $(-2 \leq n_\textup{отв.пбв}^\textup{действ} \leq +2)$, то есть $n_\textup{отв.пбв}^\textup{действ} = 1$.

Рассчитаем действительное напряжение напряжение на шинах СН в режиме наибольших нагрузок по формуле \eqref{F:U_zhelSN}:
\[
\begin{split}
U_\textup{с.нб}^\textup{действ} &= \frac{U_\textup{с.нб}^{'}\cdot U_\textup{СН.ном}}{U_\textup{ВН.ном}\cdot \left(1 + n_\textup{отв}^\textup{действ} \cdot \frac{\Delta U_\textup{отв}}{100\%}\right)} \cdot \left(1 + n_\textup{отв.пбв}^\textup{действ} \cdot \frac{\Delta U_\textup{отв.пбв}}{100\%}\right) = \\ &= \frac{107,4\cdot 38,5}{115 \cdot \left(1 - 2 \cdot \frac{1,78\%}{100\%}\right)} \left(1 + 1 \cdot \frac{2,5\%}{100\%}\right) = 38,21\; \text{кВ}
\end{split}
\]

\newpage

\section{Режим наименьших нагрузок}

\textbf{Сторона НН}

Выразим из формулы \eqref{F:U_zhelNN} ответвление регулируемой части обмотки ВН, обеспечивающее желаемое напряжение на шинах НН в режиме наименьших нагрузок:
\[n_\textup{отв}^\textup{жел} \geq \left(\frac{U_\textup{н.нм}^{'}\cdot U_\textup{НН.ном}}{U_\textup{н.нм}^\textup{жел}\cdot U_\textup{ВН.ном}} - 1\right) \cdot \frac{100\%}{\Delta U_\textup{отв}} = \left(\frac{99,6 \cdot 11}{10 \cdot 115} - 1 \right) \cdot \frac{100\%}{1,78\%} = -2,66\]

Полученное значение округляем до целого числа с учетом знака неравенства и предельного количества ответвлений $(-9 \leq n_\textup{отв}^\textup{действ} \leq +9)$, то есть $n_\textup{отв}^\textup{действ} = -2$.

Рассчитаем действительное напряжение на шинах НН в режиме наименьших нагрузок:
\[U_\textup{н.нм}^\textup{действ} = \frac{U_\textup{н.нм}^{'}\cdot U_\textup{НН.ном}}{U_\textup{ВН.ном}\cdot \left(1 + n_\textup{отв}^\textup{действ} \cdot \frac{\Delta U_\textup{отв}}{100\%}\right)} = \frac{99,6\cdot 11}{115 \cdot \left(1 - 2 \cdot \frac{1,78\%}{100\%}\right)} = 9,88\; \text{кВ}\]

\textbf{Сторона СН}

Выразим из формулы \eqref{F:U_zhelSN} ответвление регулируемой части обмотки, обеспечивающее желаемое напряжение на шинах СН в режиме наименьших нагрузок:
\[
\begin{split}
	n_\textup{отв.пбв}^\textup{жел} &= \left(\frac{U_\textup{с.нм}^\textup{жел}\cdot U_\textup{ВН.ном}\cdot \left(1 + n_\textup{отв}^\textup{действ}\cdot \frac{\Delta U_\textup{отв}}{100\%}\right)}{U_\textup{с.нм}^{'}\cdot U_\textup{СН.ном}} - 1\right) \cdot \frac{100\%}{\Delta U_\textup{отв.пбв}} = \\
	&= \left(\frac{36,8 \cdot 115 \cdot \left(1 - 2 \cdot \frac{1,78\%}{100\%}\right)}{100,0 \cdot 38,5} - 1\right) \cdot \frac{100\%}{2,5\%} = 2,40
\end{split}
\]

Полученное значение округляем до целого числа с учетом пределов ответвлений ПБВ $(-2 \leq n_\textup{отв.пбв}^\textup{действ} \leq +2)$, то есть $n_\textup{отв.пбв}^\textup{действ} = 2$.

Рассчитаем действительное напряжение на шинах СН в режиме наименьших нагрузок по формуле \eqref{F:U_zhelSN}:
\[
\begin{split}
	U_\textup{с.нм}^\textup{действ} &= \frac{U_\textup{с.нм}^{'}\cdot U_\textup{СН.ном}}{U_\textup{ВН.ном}\cdot \left(1 + n_\textup{отв}^\textup{действ} \cdot \frac{\Delta U_\textup{отв}}{100\%}\right)} \cdot \left(1 + n_\textup{отв.пбв}^\textup{действ} \cdot \frac{\Delta U_\textup{отв.пбв}}{100\%}\right) = \\ &= \frac{100,0\cdot 38,5}{115 \cdot \left(1 - 2 \cdot \frac{1,78\%}{100\%}\right)} \left(1 + 2 \cdot \frac{2,5\%}{100\%}\right) = 36,45\; \text{кВ}
\end{split}
\]

\section{Послеаварийный режим}

\textbf{Сторона НН}

Выразим из формулы \eqref{F:U_zhelNN} ответвление регулируемой части обмотки ВН, обеспечивающее желаемое напряжение на шинах НН в послеаварийном режиме:
\[n_\textup{отв}^\textup{жел} \leq \left(\frac{U_\textup{н.п/ав}^{'}\cdot U_\textup{НН.ном}}{U_\textup{н.п/ав}^\textup{жел}\cdot U_\textup{ВН.ном}} - 1\right) \cdot \frac{100\%}{\Delta U_\textup{отв}} = \left(\frac{92,3\cdot 11}{10,5\cdot 115} - 1 \right) \cdot \frac{100\%}{1,78\%} = -8,94\]

Полученное значение округляем до целого числа с учетом знака неравенства и предельного количества ответвлений $(-9 \leq n_\textup{отв}^\textup{действ} \leq +9)$, то есть $n_\textup{отв}^\textup{действ} = -9$.

Рассчитаем действительное напряжение напряжение на шинах НН в послеаварийном режиме по формуле \eqref{F:U_zhelNN}:
\[U_\textup{н.п/ав}^\textup{действ} = \frac{U_\textup{н.п/ав}^{'}\cdot U_\textup{НН.ном}}{U_\textup{ВН.ном}\cdot \left(1 + n_\textup{отв}^\textup{действ} \cdot \frac{\Delta U_\textup{отв}}{100\%}\right)} = \frac{92,3\cdot 11}{115 \cdot \left(1 - 9 \cdot \frac{1,78\%}{100\%}\right)} = 10,51\; \text{кВ}\]

\textbf{Сторона СН}

Выразим из формулы \eqref{F:U_zhelSN} ответвление регулируемой части обмотки, обеспечивающее желаемое напряжение на шинах СН в послеаварийном режиме:
\[
\begin{split}
	n_\textup{отв.пбв}^\textup{жел} &= \left(\frac{U_\textup{с.п/ав}^\textup{жел}\cdot U_\textup{ВН.ном}\cdot \left(1 + n_\textup{отв}^\textup{действ}\cdot \frac{\Delta U_\textup{отв}}{100\%}\right)}{U_\textup{с.п/ав}^{'}\cdot U_\textup{СН.ном}} - 1\right) \cdot \frac{100\%}{\Delta U_\textup{отв.пбв}} = \\
	&= \left(\frac{38,5 \cdot 115 \cdot \left(1 - 9 \cdot \frac{1,78\%}{100\%}\right)}{93,5 \cdot 38,5} - 1\right) \cdot \frac{100\%}{2,5\%} = 1,32
\end{split}
\]

Полученное значение округляем до целого числа с учетом пределов ответвлений ПБВ $(-2 \leq n_\textup{отв.пбв}^\textup{действ} \leq +2)$, то есть $n_\textup{отв.пбв}^\textup{действ} = 1$

Рассчитаем действительное напряжение напряжение на шинах СН в послеаварийном режиме по формуле \eqref{F:U_zhelSN}:
\[
\begin{split}
	U_\textup{с.п/ав}^\textup{действ} &= \frac{U_\textup{с.п/ав}^{'}\cdot U_\textup{СН.ном}}{U_\textup{ВН.ном}\cdot \left(1 + n_\textup{отв}^\textup{действ} \cdot \frac{\Delta U_\textup{отв}}{100\%}\right)} \cdot \left(1 + n_\textup{отв.пбв}^\textup{действ} \cdot \frac{\Delta U_\textup{отв.пбв}}{100\%}\right) = \\ &= \frac{93,5\cdot 38,5}{115 \cdot \left(1 - 9 \cdot \frac{1,78\%}{100\%}\right)} \left(1 + 1 \cdot \frac{2,5\%}{100\%}\right) = 38,21\; \text{кВ}
\end{split}
\]

В итоге можно сделать вывод о достаточности регулировочных диапазонов устройств РПН и ПБВ для поддержания желаемого уровня напряжения на шинах НН и СН ПС во всех режимах.