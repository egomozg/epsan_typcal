\Introduction

Целью выполнения расчетного задания является получение практических навыков по решению задач, связанных с составлением схем замещения основных элементов электрических сетей, к которым относятся линии электропередачи (ЛЭП) и подстанции (ПС), а также расчетам установившихся нормальных и послеаварийных режимов электрической сети и анализу полученных результатов.

Кроме того, в рамках расчетного задания решаются задачи оценки достаточности регулировочных диапазонов устройств регулирования напряжения трансформаторов на понижающей подстанции в различных режимах работы электрической сети, рассчитываются потери активной мощности и годовые потери электроэнергии в сети.

Полученные практические навыки необходимы для выполнения предстоящего курсового проекта по дисциплине «Электроэнергетические системы и сети» и будут полезны при подготовке выпускной квалификационной работы бакалавра.
